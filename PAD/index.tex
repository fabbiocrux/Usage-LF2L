\hypertarget{usage-puxe9dagogique-du-lf2l}{%
\subsubsection{Usage Pédagogique du
LF2L}\label{usage-puxe9dagogique-du-lf2l}}

Les graphiques suivant montrent approximativement l'usage du LF2L dans
le cadre pédagogique.

A noter que le LF2L est ouvert environ 1840 / an, à partir du calcul
suivant : 40h / semaine X 46 semaines sur la base d'une ouverture
9h-12h30 et 13h30-18h, sachant que nous ouvrons avant 9h et que c'est
hors événements spéciaux + NYBI (19h - Minuit tous les jeudis)

soit 7 360 pour 4 ans (2015-2018) et sachant que deux, voire trois
activités pédagogiques peuvent avoir lieu en même temps : par ex - un
atelier côté espace collaboratif + un cours côté matérialsiation et un
groupe projet 2AI\ldots{}

Nous n'avons ici pas référencé tous les usages de autres composantes ou
services centraux de l'UL ou de l'ICEEL En effet ces données sont très
éparses (une visite ou un ateliers de temps en temps) - estimation : 30
à 40 h / an

\includegraphics{/uploads/upload_a23937aab238f9d5b686efbffb326cb0.png}

Commentaire sur ces données brutes (cumul sur 4 années): - les tendances
des usages se retrouve sur le 2ème graphique

\begin{itemize}
\item
  l'utilisation pour Telecom est en baisse constante : très peu en 2018,
  aucune en 2019

  \begin{itemize}
  \tightlist
  \item
    démonstrateurs 2015-2018 produits dans ce cadre :
    http://iamd-mom.telecomnancy.univ-lorraine.fr/
  \item
    en 2019 nous explorons la dimension Big Data du GID Route
  \end{itemize}
\item
  la plupart des heures IUVTT (ou EDUTER) sont liés à des ateliers
  projets sur des sujets en lien avec des projets de recherche ERPI (ANR
  ville durable, Diaclimap, GID Route, etc.) ou stratégie de site
  ENSGSI-ERPI-LF2L (ex: atelier Octroi).
\item
  Les heures Ingexys pétale 1 correspondent aux Ateliers d'Innovation
  Urbaine réalisé dans le cadre de projets (ex: Smart city
  Alzette-Belval pour les 3 dernières éditions) rassemblant plusieurs
  masters (60 à 80 étudiants selon les années) avec une nouvelle
  répartition depuis la rentrée 2018

  \begin{itemize}
  \tightlist
  \item
    Urbansime et aménagement : 2 parcours (IUVTT + Intelligence
    territoriale à Metz)
  \item
    Proj\&Ter (Staps) : 1 parcours
  \item
    Master Formation, travail et territoire en développement
  \item
    Géographie (à Nancy) : parcours (a priori) (dans le futur voir s'il
    pourrait être intéressant d'associer les parcours type
    Sol-Environnement ?)
  \end{itemize}
\item
  Les heures Ingexys pétale 2 rassemblent depuis la rentrée 2018:

  \begin{itemize}
  \tightlist
  \item
    IDEAS
  \item
    ISC (NB aucun étudiants n'a participé cette année) (avant 2018, il y
    avait aussi SPIEQ de L. Perrin, on a aussi eu EDA et IDE)
  \end{itemize}
\item
  les heures ENSGSI : cours de CAO, maquettage (Alaa + Patrick), Projets
  1AI ou 2AI, nouveau Cours d'Hakim, Soutenances CI10, Module de
  Mauricio et Olivier P., etc.
\item
  EDUTER : ateliers sur projet
\end{itemize}

\includegraphics{/uploads/upload_6ccd776ee20fa364e91d74b59c8c718d.png}
